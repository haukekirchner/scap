%pdflatex --shell-escape scap_benchmark_dl_training
%biber scap_benchmark_dl_training 

\PassOptionsToPackage{hyphens}{url}
\documentclass[compress,aspectratio=169]{beamer}

\usepackage[official]{eurosym}
\usepackage{multirow}
\setlength{\marginparwidth}{2cm}
\usepackage{todonotes}
\presetkeys{todonotes}{inline}{}
\usepackage[style=verbose,backend=biber,style=authoryear, citestyle=authoryear ]{biblatex}
\setbeamertemplate{bibliography item}{\insertbiblabel} % Ensure that cite labels appear in references section
\addbibresource{ref.bib}
\usepackage{./assets/beamerthemeGoettingen} % make sure the theme file is on this path
\graphicspath{{../}{./assets/}}
\usepackage{caption}
\usepackage{booktabs}

\hypersetup{
    colorlinks,
    citecolor=black,
}

\newcommand{\source}[1]{\par\begin{textblock*}{6cm}(1cm,8cm)
    \begin{beamercolorbox}[wd=\paperwidth,ht=0.5cm,right]{framesource}
        \usebeamerfont{framesource}\usebeamercolor[fg]{framesource} \centering\tiny {#1}
    \end{beamercolorbox}
\end{textblock*}}


% listing / code
\usepackage{minted}
\usemintedstyle{tango}
% Box listing / code
\usepackage{tcolorbox}

% Box listing / code style 
% These options will be applied to all `tcolorboxes`
\tcbset{%
    noparskip,
    colback=gray!5, %background color of the box
    colframe=gray!20, %color of frame and title background
    coltext=black, %color of body text
    coltitle=black, %color of title text 
    fonttitle=\tiny,
    alerted/.style={coltitle=red, 
                     colframe=gray!40},
    example/.style={coltitle=black, 
                     colframe=green!20,             
                     colback=green!5},
    }

\lstset{literate=%
    {Ö}{{\"O}}1
    {Ä}{{\"A}}1
    {Ü}{{\"U}}1
    {ß}{{\ss}}1
    {ü}{{\"u}}1
    {ä}{{\"a}}1
    {ö}{{\"o}}1
    {~}{{\textasciitilde}}1
}

\usepackage{csquotes} % For \enqoute{}
\usepackage{hyperref}

% --- document configuration ---
\newcommand{\mytitle}{Getting started with profiling PyTorch}     
% Leave empty for no subtitle
\newcommand{\mysubtitle}{State of the art and plan for Scalable Computing Systems and Applications in AI, Big Data and HPC}   
\newcommand{\myauthor}{Hauke Kirchner}
\newcommand{\myauthorurl}{\href{http://www.overleaf.com}{Something 
Linky}}
\newcommand{\myvenue}{Göttingen}
% For example, use \today
\newcommand{\mydate}{24.11.2022}
% For example, Institute for Computer Science / GWDG
\newcommand{\myinstitute}{GWDG - AG Computing}
% Leave empty for no footer image
\newcommand{\myfooterimage}{}           
\newcommand{\mygrouplogo}{}
% Images must be enabled manually under title page \titleLogo
% Adjust position and width manually for fewer images
\newcommand{\mytitleimageone}{}         
\newcommand{\mytitleimagetwo}{}        
\newcommand{\mytitleimagethree}{}

% --- title page ---
\title{\Large \mytitle}
\venue{\myvenue}
\date{\mydate}
\subtitle{\mysubtitle}
%\authorURL{\myauthorurl}
\author{{\myauthor}}
\authorFooter{\myauthor \hspace{0.3cm} \includegraphics[height=1em]{\myfooterimage}}
\institute{\myinstitute}
%\groupLogo{\includegraphics[width=2cm]{\mygrouplogo}}
\titleLogo{
%\includegraphics[height=2.7cm]{\mytitleimageone}
%\includegraphics[height=2.7cm]{\mytitleimagetwo}
%\includegraphics[height=2.7cm]{\mytitleimagethree}
}

\setbeamertemplate{footline}[text line]{
\begin{beamercolorbox}[sep=0.5em,wd=\paperwidth,leftskip=0.2cm,rightskip=0.1cm]{footlinecolor}
\myauthor \hfill \insertVenue \hfill \insertframenumber\,/\,\ref{pg:lastpage}
\end{beamercolorbox}
}

\begin{document}

\begin{frame}[plain]
	\titlepage
\end{frame}

\begin{frame}[t]{Table of contents}
  \tableofcontents[subsectionstyle=hide/hide]
\end{frame}

% --- slides begin ---

\section{Motivation}

\begin{frame}{Why is it important to profile the training process of neural networks?}

    \begin{columns}
        \begin{column}{0.5\textwidth}
            \begin{block}{\centering Training speed}
                \centering
                \vspace{3em}
                \includegraphics[width=0.3\textwidth]{assets/speed_FILL0_wght400_GRAD0_opsz48.png}
            \end{block}
        \end{column}
        \begin{column}{0.5\textwidth}
            \begin{block}{\centering Energy efficiency}
                \centering
                \vspace{3em}
                \includegraphics[width=0.3\textwidth]{assets/electric_bolt_FILL0_wght400_GRAD0_opsz48}
            \end{block}
        \end{column}
    \end{columns}

\end{frame}

\begin{frame}{Training speed 
              \begin{tabular}{@{}c@{}}
                  \includegraphics[width=0.05\textwidth]{assets/speed_FILL0_wght400_GRAD0_opsz48.png}
              \end{tabular}
              }

    \vspace{-3em}

    \begin{columns}
        \begin{column}{0.6\textwidth}
            \begin{itemize}
                \item training of neural networks is computationally intensive
                \item[$\hookrightarrow$] workflow needs to be optimized for available hardware
                    \begin{itemize}
                        \item How to make good use of clusters with heterogenous hardware?
                        \item How many GPUs are worth requesting?
                    \end{itemize}
                    \vspace{2em}
                \item[$\Rightarrow$] optimization for available accelerators is critical in ML/DL
            \end{itemize}
        \end{column}
        \begin{column}{0.4\textwidth}
            \vspace{-1em}
            \begin{figure}
                \includegraphics[width=0.9\textwidth]{assets/gwdg_scc.png}
                \caption*{Structure and resources of a part of the Scientific Compute Cluster.}
            \end{figure}
        \end{column}
    \end{columns}

    \source{Image source: Adapted from \url{https://www.gwdg.de/web/guest/hpc-on-campus/scc}, accessed on: 09.11.2022}
\end{frame}

\begin{frame}{Energy efficiency 
              \begin{tabular}{@{}c@{}}
                  \includegraphics[width=0.05\textwidth]{assets/electric_bolt_FILL0_wght400_GRAD0_opsz48}
              \end{tabular}
              }



    \begin{columns}
        
        \begin{column}{0.6\textwidth}
            \begin{itemize}
                \item idea of \textbf{GreenAI}~(\cite{schwartz_2019_greenai})\\[1em]
                \item very different energy requirements (finetuning vs training)\\[1em]
                \item deep learning is emerging in several fields
                    \begin{itemize}
                        \item[$\Rightarrow$] the impact on energy consumption and consequently, our climate is growing
                    \end{itemize}
            \end{itemize}
        \end{column}

        \begin{column}{0.4\textwidth}
            \vspace{-3.5em}
            \centering
            \begin{figure}
            \includegraphics[width=\textwidth]{assets/20220610_dodge_measuring-the-carbon-intensity-of-ai-in-cloud-instances-fig2-bert.png}
            \caption*{$CO_2$ Relative Size Comparison}
            \end{figure}
            \source{Image source: Adapted from \cite{20220610_dodge_measuring-the-carbon-intensity-of-ai-in-cloud-instances}}
            
        \end{column}
    \end{columns}



\end{frame}

\section{Methods}
\sectionIntroHidden % Show an outline of the current section with hidden subsections
%\sectionIntro % Show an outline of the current section with subsections

\begin{frame}{Metrics}

% Please add the following required packages to your document preamble:
% \usepackage{booktabs}
% \usepackage{multirow}
\begin{table}[]
\begin{tabular}{@{}ll@{}}
\toprule
metric                                                                                 & purpose                        \\ \midrule
Execution time                                                                         & \multirow{2}{*}{traditionally} \\
FLOPS                                                                                  &                                \\ \midrule
Throughput: $\frac{images}{sec}$                                                       & with the advent of GPUs        \\ \midrule
Time to Accuracy (TTA)                                                                 & \multirow{2}{*}{deep learning} \\
\begin{tabular}[c]{@{}l@{}}Average Time to \\ Multiple Thresholds (ATTMT)\end{tabular} &                                \\ \bottomrule
\end{tabular}
\end{table}


\source{Image source: Adapted from \url{https://snehilverma41.github.io/Metrics_ML_FastPath19.pdf}}

\end{frame}


\section{Tool overview}
\sectionIntroHidden

\begin{frame}[t]{Tools}
    \begin{center}
        \includegraphics[width=1\textwidth]{./assets/tools}
    \end{center}
    \begin{columns}[t]
    \begin{column}{0.5\textwidth}
                \begin{itemize}
                    \item PyTorch - Profiler~\footnote{\tiny{\url{https://pytorch.org/tutorials/intermediate/tensorboard_profiler_tutorial.html}}}
                    \item collection of performance metrics
                    \item indentification of\\expensive operators
                    \item tracking of the kernel activity
                \end{itemize}
    \end{column}
    \begin{column}{0.5\textwidth}
                \begin{itemize}
                    \item FlopsProfiler~\footnote{\tiny{\url{https://www.deepspeed.ai/tutorials/flops-profiler/}}}
                    \item model speed (latency, throughput)
                    \item efficiency (FLOPS\footnote{\tiny{floating-point operations per second}})
                \end{itemize}
    \end{column}
    \end{columns}
\end{frame}

\section{PyTorch Profiler}
\sectionIntroHidden

\begin{frame}{PyTorch Profiler With TensorBoard}
    \vspace{-1em}
    \begin{center}
    \begin{figure}
        \includegraphics[width=1\textwidth]{assets/profiler_overview1}
    \end{figure}
    \end{center}
\source{Image source: \url{https://pytorch.org/tutorials/intermediate/tensorboard_profiler_tutorial.html}, accessed on: 22.11.2022}

\end{frame}

\begin{frame}{PyTorch Profiler With TensorBoard}
    \vspace{-1em}
    \begin{center}
    \begin{figure}
        \includegraphics[width=1\textwidth]{assets/profiler_overview_gpu_summary}
    \end{figure}
    \end{center}
\source{Image source: Adapted from \url{https://pytorch.org/tutorials/intermediate/tensorboard_profiler_tutorial.html}, accessed on: 22.11.2022}

\end{frame}

\begin{frame}{PyTorch Profiler With TensorBoard}
    \vspace{-1em}
    \begin{center}
    \begin{figure}
        \includegraphics[width=1\textwidth]{assets/profiler_overview_execution_time}
    \end{figure}
    \end{center}
\source{Image source: Adapted from \url{https://pytorch.org/tutorials/intermediate/tensorboard_profiler_tutorial.html}, accessed on: 22.11.2022}

\end{frame}

\begin{frame}{PyTorch Profiler With TensorBoard}
    \vspace{-1em}
    \begin{center}
    \begin{figure}
        \includegraphics[width=1\textwidth]{assets/profiler_overview_step_time_breakdown}
    \end{figure}
    \end{center}
\source{Image source: Adapted from \url{https://pytorch.org/tutorials/intermediate/tensorboard_profiler_tutorial.html}, accessed on: 22.11.2022}

\end{frame}

\end{frame}

\begin{frame}{PyTorch Profiler With TensorBoard}
    \vspace{-1em}
    \begin{center}
    \begin{figure}
        \includegraphics[width=1\textwidth]{assets/profiler_overview_performance_reccomondation}
    \end{figure}
    \end{center}
\source{Image source: Adapted from \url{https://pytorch.org/tutorials/intermediate/tensorboard_profiler_tutorial.html}, accessed on: 22.11.2022}

\end{frame}

\section{DeepSpeed - FlopsProfiler}
%\sectionIntroHidden % Show an outline of the current section with hidden subsections
\sectionIntro % Show an outline of the current section with subsections

\begin{frame}[fragile]{Summary}
    \vspace{-1em}
    \footnotesize\inputminted[xleftmargin=1em,linenos,fontsize=\scriptsize, firstline=1,lastline=20]{python}{./assets/deepspeed-flopsprofiler_example.txt }

\end{frame}

\begin{frame}[fragile]{Aggregated Profile per GPU}
    \vspace{-1em}
    \footnotesize\inputminted[xleftmargin=1em,linenos,fontsize=\scriptsize, firstline=28,lastline=46]{python}{./assets/deepspeed-flopsprofiler_example.txt }
    }

\end{frame}

\begin{frame}[fragile]{Detailed Profile per GPU}

    \footnotesize\inputminted[xleftmargin=1em,linenos,fontsize=\scriptsize, firstline=59,lastline=72, breaklines]{python}{./assets/deepspeed-flopsprofiler_example.txt }
    }

\end{frame}

\begin{frame}[fragile]{Torch profiler}
    \footnotesize\inputminted[xleftmargin=1em,linenos,fontsize=\scriptsize, highlightlines={4-10,14,15}]{python}{./assets/profiler-torch.py}

    \source{Tutorial: \url{https://pytorch.org/tutorials/intermediate/tensorboard_profiler_tutorial.html}, accessed on: 24.03.2023}
\end{frame}

\begin{frame}[fragile]{Deepspeed - FlopsProfiler}
    \vspace{-1em}
    \footnotesize\inputminted[xleftmargin=1em,linenos,fontsize=\scriptsize, highlightlines={1,3,4,8,9,11-18}]{python}{./assets/deepspeed.py}

    \source{Tutorial: \url{https://www.deepspeed.ai/tutorials/flops-profiler/}, accessed on: 24.03.2023}
\end{frame}


\begin{frame}{Profiling is the first step of optimizing}

\begin{columns}
        \begin{column}{0.5\textwidth}
            \centering
            \vspace{-1em}
            \begin{figure}
            \includegraphics[width=0.85\textwidth]{assets/Sherlock-Holmes-locates-the-best-graphical-processing-unit-inside-the-data-center-for-his-deep-learning-workflow}
            \caption*{Image generated with stable diffusion: \\
            \tiny{"Sherlock Holmes locates the best graphical processing unit inside the data center for his deep learning workflow"}}
            \end{figure}
        \end{column}
        \begin{column}{0.5\textwidth}
            \begin{itemize}
                \item "Stable Diffusion v1 version of the model requires 150,000 A100 GPU Hours for a single training session"\footnote{\tiny{\url{https://syncedreview.com/2022/11/09/almost-7x-cheaper-colossal-ais-open-source-solution-accelerates-aigc-at-a-low-cost-diffusion-pretraining-and-hardware-fine-tuning-can-be/}}, accessed on: 10.11.2022}
                \vspace{1em}
                \item[$\Rightarrow$] Optimization of deep learning workflows is of growing importance for energy efficiency.
            \end{itemize}
        \end{column}
    \end{columns}

\end{frame}

\begin{frame}
\label{pg:lastpage} % Label on last frame to get the page number for footer
\centering
Try it yourself!
\end{frame}

\begin{frame}{References}
    % References slide in appendix
    \renewcommand*{\bibfont}{\normalfont\scriptsize}
    \printbibliography[heading=none]
\end{frame}

\end{document}
