\documentclass[12pt, a4paper, hidelinks]{article}

\usepackage{graphicx}
\usepackage{hyperref}
\usepackage{xcolor}
\usepackage[english]{babel}
\usepackage[nottoc]{tocbibind}
\usepackage[utf8]{inputenc}
\usepackage[T1]{fontenc}
\usepackage[style=verbose,backend=biber,style=authoryear, citestyle=authoryear ]{biblatex}
\addbibresource{ref.bib} % The filename of the bibliography
\usepackage[left=2.5cm,right=2.5cm,top=2.5cm,bottom=2.5cm]{geometry}
\usepackage{acronym}
\usepackage{fancyhdr}
\pagestyle{fancy}
\usepackage{lipsum}
\usepackage{array, tabularx, booktabs} % For better tables
\usepackage{booktabs}
\usepackage{multirow}
\usepackage{longtable}
%\usepackage{listings} % Code environments
%\usepackage{lstautogobble}  % Fix relative indenting
%\usepackage{color}          % Code coloring
%\usepackage{zi4}            % Nice font
\usepackage{minted}
\usemintedstyle{tango}
\setminted{linenos, autogobble, bgcolor=gray!5}
\usepackage{cleveref}
\usepackage{titlesec}
\setlength{\marginparwidth}{2cm}
\usepackage{todonotes}
\usepackage[autostyle=true]{csquotes}

\iffalse

\definecolor{codegreen}{rgb}{0,0.6,0}
\definecolor{codegray}{rgb}{0.5,0.5,0.5}
\definecolor{backcolour}{gray}{0.95}
\definecolor{codeorange}{rgb}{0.8,0.5,0.2}

\lstdefinestyle{mystyle}{
    backgroundcolor=\color{backcolour},  
    commentstyle=\color{codegray},
    keywordstyle=\color{codeorange},
    numberstyle=\tiny\color{codegray},
    stringstyle=\color{codegreen},
    basicstyle=\ttfamily\footnotesize,
    breakatwhitespace=false,         
    breaklines=true,                 
    captionpos=b,                    
    keepspaces=true,                 
    numbers=left,                    
    numbersep=5pt,                  
    showspaces=false,                
    showstringspaces=false,
    showtabs=false,                  
    tabsize=2
}

%\lstset{style=mystyle}

\definecolor{bluekeywords}{rgb}{0.13, 0.13, 1}
\definecolor{greencomments}{rgb}{0, 0.5, 0}
\definecolor{redstrings}{rgb}{0.9, 0, 0}
\definecolor{graynumbers}{rgb}{0.5, 0.5, 0.5}

\usepackage{listings}
\lstset{
    autogobble,
    columns=fullflexible,
    showspaces=false,
    showtabs=false,
    breaklines=true,
    showstringspaces=false,
    breakatwhitespace=true,
    escapeinside={(*@}{@*)},
    commentstyle=\color{greencomments},
    keywordstyle=\color{bluekeywords},
    stringstyle=\color{redstrings},
    numberstyle=\color{graynumbers},
    basicstyle=\ttfamily\footnotesize,
    frame=l,
    framesep=12pt,
    xleftmargin=12pt,
    tabsize=4,
    captionpos=b
}

\fi

\titleformat{\section}[hang]{\huge\bfseries}{\thesection\hspace{20pt}}{0pt}{\huge\bfseries}
\graphicspath{{Figures/}{./}{Assets/}}

% --- document configuration ---

\newcommand{\thesistitle}{Profiling tree species classification with synthetic data and deep learning}  
\newcommand{\supervisor}{Dorothea Sommer} 
\newcommand{\authorname}{Hauke Kirchner}
%\newcommand{\university}{Georg-August-Universität Göttingen}
%\newcommand{\department}{Institute of Computer Science}
\newcommand{\thesistype}{Seminar Report}
%\newcommand{\matrikelnumber}{[Put your matrikel number here]}
\newcommand{\keywords}{} % Set keywords that describe your report

\hypersetup{pdftitle=\thesistitle} % Set the PDF's title to your title
\hypersetup{pdfauthor=\authorname} % Set the PDF's author to your name
\hypersetup{pdfkeywords=\keywords} % Set the PDF's keywords to your keywords

\begin{document}

\fancyhead{}
\fancyhead[R]{\footnotesize \thesistitle}
\fancyfoot{}
\fancyfoot[R]{\thepage}
\fancyfoot[L]{Section \thesection}
\fancyfoot[C]{\authorname}
\renewcommand{\headrulewidth}{0.4pt}
\renewcommand{\footrulewidth}{0.4pt}

\pagestyle{plain}


    

% --- title page ---

\begin{titlepage}
\begin{minipage}[t]{0.6\textwidth}
\begin{flushleft}
\includegraphics[width=6.5cm]{logo-goettingen.pdf}
\end{flushleft}
\end{minipage}
\begin{minipage}[t]{0.4\textwidth}
\begin{center}
\qquad\includegraphics[width=2.5cm]{hps-logo.pdf}
\end{center}
\end{minipage}

\begin{center}

\vspace*{.06\textheight}
\LARGE \thesistype\\[0.5cm]

\rule{.9\linewidth}{.6pt} \\[0.4cm] % Horizontal line
{\huge \bfseries \thesistitle}\vspace{0.4cm}
\rule{.9\linewidth}{.6pt} \\[1.5cm] % Horizontal line
 
\Large\authorname\\
\hfill\\
%\large MatrNr: \matrikelnumber\\ \vfill
Supervisor: \supervisor 
\vfill
%\university\\
%\department
\vfill
{\large \today}\\[4cm] % Date
 
\vfill
\end{center}
\end{titlepage}
    

% --- abstract ---

%\thispagestyle{empty}
\newpage
\pagenumbering{roman}
\setcounter{page}{1}

\section*{Abstract}


\medskip
\iffalse
General structure of an abstract, write 1 to 2 sentences per section
\begin{enumerate}
\item  A general statement introducing the broad research area of the particular topic being investigated.
\item  An explanation of the specific problem (difficulty, obstacle, challenge) to be solved.
\item  A review of existing or standard solutions to this problem and their limitations.
\item  An outline of the proposed new solution.
\item  A summary of how the solution was evaluated and what the outcomes of the evaluation were.
\end{enumerate}

You may find the following resources useful:
\begin{itemize}
    \item \url{https://www.grammarly.com/blog/write-an-abstract/}
    \item \url{https://www.editage.com/insights/manuscript-structure-how-to-convey-your-most-important-ideas-through-your-paper}
    \item More useful links: \url{https://hps.vi4io.org/teaching/ressources/start}
\end{itemize}
\fi

%\thispagestyle{empty}
% --- table of contents ---
% Comment out the lists you are not using
\newpage


\clearpage
\phantomsection\pdfbookmark{\contentsname}{toc}
\tableofcontents

\newpage
\clearpage\phantomsection
\listoftables
%\newpage
%\clearpage
\phantomsection
\listoffigures
%\newpage
%\clearpage
\phantomsection
\listoflistings \addcontentsline{toc}{section}{List of Listings}

% if you have abbreviations
\newpage

\section*{List of Abbreviations} \addcontentsline{toc}{section}{List of Abbreviations}
\begin{acronym}[Bash] % Add acronyms such that they are shown in full only on first occurrence
    \acro{HPC}{High-Performance Computing}
\end{acronym}

\thispagestyle{plain}
\newpage

% --- content ---

\pagenumbering{arabic}
\setcounter{page}{1}
\pagestyle{fancy}


\section{Introduction}

\iffalse
Why is it important to benchmark the training process of neural networks?

1. Training speed
2. Energy efficiency
\fi

Recent advances in deep learning made the technology more popular in several fields of research.
This lead to breakthroughs which helped to ...
Increasingly applications in forestry profit from deep learning applications and pubications in recent years have shown the great potential. Often traning data is a bottleneck for the application of neural networks. Especially in the field of environmental research collecting data requires substential effort. In Germany the largest effort to cellect data of German forests is the "Bundeswaldinventur". This involves the work of multiple teams over 

Especilly for lidar data of forests not to many pre-trained networks exist. Therfore, we test an approach where the pre-training is done based on synthetically generated lidar data representing individual trees.

Simultaniously, the emerging use of deep learning also leads to an increase in energy consumption. This phenomen studied by \cite{20220610_dodge_measuring-the-carbon-intensity-of-ai-in-cloud-instances}. 

\begin{itemize}
    \item Identify \textbf{profiling tools} that can help to \textbf{optimize an existing PyTorch workflow}.
    \item As this approach should help scientists, the \textbf{usibility is of high importance}. The simplest tool for doing the job is preferred.
    \item In contrast to training benchmark suites, such as MLPerf, I do not focus on benchmarking hardware but on optimizing an existing PyTorch workflow (viewpoint of a scientist)
\end{itemize}

\iffalse
The introduction is the most important part of such a report. Its general structure is similar to the Abstract but with about one paragraph per section as listed again below.
\begin{enumerate}
\item  A general statement introducing the broad research area of the particular topic being investigated.
\item  An explanation of the specific problem (difficulty, obstacle, challenge) to be solved.
\item  A review of existing or standard solutions to this problem and their limitations.
\item  An outline of the proposed new solution.
\item  A summary of how the solution was evaluated and what the outcomes of the evaluation were.
\end{enumerate}
Furthermore, the introduction should have a contributions section, that summarizes, possibly as a list, what the report and the work described in the report has contributed.
Finally, the introduction should end with an outline of the remaining report.

\ac{HPC} refers to the usage of powerful compute systems to solve non-trivial problems.

\subsection{Citation and figure example}
Hawthorn et al. \cite{20220610_dodge_measuring-the-carbon-intensity-of-ai-in-cloud-instances} talk about laser, which is similar to the work of his colleagues \cite{20220610_dodge_measuring-the-carbon-intensity-of-ai-in-cloud-instances, 20220610_dodge_measuring-the-carbon-intensity-of-ai-in-cloud-instances}.
\begin{figure}[th]
\centering
\includegraphics[width=0.7\textwidth]{speed_FILL0_wght400_GRAD0_opsz48.png}
\caption[An Electron]{An electron (artist's impression).}
\label{fig:Electron}
\end{figure}

\subsection{Table and listing example}
\Cref{tab:treatments} shows an example for a table in \LaTeX.
\begin{table}[th]
\label{tab:treatments}
\centering
\begin{tabularx}{0.45\textwidth}{l||r|r}
Groups & Treatment X & Treatment Y \\
\hline \hline
1 & 0.20 & 0.80\\
2 & 0.17 & 0.70\\
3 & 0.24 & 0.75\\
4 & 0.68 & 0.30\\
\end{tabularx}
\caption{The effects of treatments X and Y on the four groups studied.}
\end{table}

[If you show results in tables, always ensure all of them are aligned to the right and have the same precision.]

\Cref{lst:hello} shows an example of a listing, a good way to display code in \LaTeX.

\begin{listing}
\begin{minted}{Go}
package main
import "fmt"
func main() {
    fmt.Println("Hello, world!")
}
\end{minted}
\caption{"Hello, world!" in Go}
\label{lst:hello}
\end{listing}
\fi

\iffalse
\begin{lstlisting}[language=Go, caption={"Hello, world!" in Go}, label=lst:hello]
package main
import "fmt"
func main() {
    fmt.Println("Hello, world!")
}
\end{lstlisting}
\fi

\section{Methods}
\label{sec:methods}

\subsection{Tree species classification based on synthetic lidar data}
\label{sec:workflow}

\cite{white_2016_als-forest-inventory}
\cite{uba_2020_waldumbau}
\cite{weber_1995_arbaro}
Arbaro implementation \footnote{\url{https://github.com/wdiestel/arbaro} (Accessed on: 22.02.2023)}
\cite{pointnet}
\cite{verma_2019_metrics-ml-benchmarking}
pytorch tutorial \footnote{\url{https://pytorch.org/tutorials/intermediate/tensorboard_profiler_tutorial.html}, accessed on: 22.11.2022}

\begin{figure}[H]
\centering
\includegraphics[width=0.4\textwidth]{assets/beech_drought}
\caption*{Spruce and beech stand close to the national park "Hainich".}
\label{fig:uav-hainich}
\end{figure}


\begin{figure}[H]
\centering
\includegraphics[width=0.5\textwidth]{assets/workflow_synthetic_data}
\caption[Synthetic- lidar workflow]{This workflow shows ... Adapted from \cite{9906068}}
\label{fig:lidar-workflow}
\end{figure}

\subsection{Fundamentals of software profiling}
\label{sec:profiling}

% Please add the following required packages to your document preamble:
% \usepackage{booktabs}
% \usepackage{multirow}
\begin{table}[H]
\label{tab:metrics}
\centering
\begin{tabular}{@{}ll@{}}
\toprule
metric                                                                                 & purpose                        \\ \midrule
Execution time                                                                         & \multirow{2}{*}{traditionally} \\
FLOPS                                                                                  &                                \\ \midrule
Throughput: $\frac{images}{sec}$                                                       & with the advent of GPUs        \\ \bottomrule
\end{tabular}
\caption{Adapted from \url{https://snehilverma41.github.io/Metrics_ML_FastPath19.pdf}}
\end{table}


!!!Test reference: \Cref{tab:metrics}

\subsection{Tools Overview}
\label{sec:tools-overview}

% Please add the following required packages to your document preamble:
% \usepackage{booktabs}
% \usepackage{multirow}
\begin{table}[H]
\label{tab:tools}
\centering
\begin{tabular}{@{}lll@{}}
\toprule
tool                              & metrics                                                                                & scope                    \\ \midrule
\href{https://pytorch.org/tutorials/intermediate/tensorboard_profiler_tutorial.html}{PyTorch Profiler With TensorBoard} & \begin{tabular}[c]{@{}l@{}}performance metrics\\ (e.g. time, memory)\end{tabular}      & \multirow{2}{*}{PyTorch} \\
\href{https://github.com/facebookresearch/fvcore}{fvcore}                      & FLOPS                                                                                  &                          \\ \midrule
\href{https://www.intel.com/content/www/us/en/developer/tools/oneapi/vtune-profiler.html}{Vtune}                             & \multirow{2}{*}{\begin{tabular}[c]{@{}l@{}}general\\ performance metrics\end{tabular}} & Intel-only               \\
\href{https://github.com/RRZE-HPC/likwid}{likwid}                            &                                                                                        & general                  \\ \bottomrule
\end{tabular}
\caption[test caption new]{test caption}
\end{table}



\begin{itemize}
    \item[$\Rightarrow$] different tools for different use cases
    \item[$\Rightarrow$] Vtune and likwid can profile all kind of applications
    \item[$\Rightarrow$] here I will focus on profiling tools optimized for PyTorch
\end{itemize}

\subsection{Tensorboard}
\label{sec:Tensorboard}

\begin{listing}[H]
\inputminted[xleftmargin=1em,linenos,fontsize=\scriptsize, highlightlines={1,3,9-12,14-16}]{python}{./assets/tensorboard.py}
\caption{}
\label{lst:tensorboard}
\end{listing}

\subsection{PyTorch - Profiler}
\label{sec:m-pytorch-profiler}

\begin{listing}[H]
\inputminted[xleftmargin=1em,linenos,fontsize=\scriptsize, highlightlines={4-10,14,15}]{python}{./assets/profiler-torch.py}
\caption{}
\label{lst:profiler-torch}
\end{listing}

\subsection{Deepspeed - FlopsProfiler}
\label{sec:m-flopsprofiler}


\begin{listing}[H]
\inputminted[xleftmargin=1em,linenos,fontsize=\scriptsize, highlightlines={1,3,4,8,9,11-18}]{python}{./assets/deepspeed.py}
\caption{}
\label{lst:deepspeed}
\end{listing}

\subsection{Experiments}
\label{sec:m-experiments}

\begin{figure}[H]
\centering
\includegraphics[width=0.5\textwidth]{./assets/experiments.png}
\caption*{}
\label{fig:experiments}
\end{figure}

\begin{table}[]
\label{tab:experiments-all}
\begin{tabular}{@{}llllll@{}}
\toprule
run & node         & tool           & job\_id  & is\_valid & experiment     \\ \midrule
1   & scc\_cpu     & no-tool        & 14629421 & TRUE      &                \\
2   & scc\_cpu     & tensorboard    & 14629426 & TRUE      &                \\
3   & scc\_cpu     & profiler-torch & 14650740 & TRUE      &                \\
4   & scc\_cpu     & deepspeed      & 14617521 & FALSE     &                \\
5   & scc\_gtx1080 & no-tool        & 14619617 & TRUE      &                \\
6   & scc\_gtx1080 & tensorboard    & 14615343 & TRUE      &                \\
7   & scc\_gtx1080 & profiler-torch & 14650076 & TRUE      &                \\
8   & scc\_gtx1080 & deepspeed      & 14615344 & TRUE      &                \\
9   & scc\_rtx5000 & no-tool        & 14619618 & TRUE      &                \\
10  & scc\_rtx5000 & tensorboard    & 14617172 & TRUE      &                \\
11  & scc\_rtx5000 & profiler-torch & 14650079 & TRUE      &                \\
12  & scc\_rtx5000 & deepspeed      & 14617171 & TRUE      &                \\
13  & scc\_v100    & no-tool        & 14619619 & TRUE      &                \\
14  & scc\_v100    & tensorboard    & 14617203 & TRUE      &                \\
15  & scc\_v100    & profiler-torch & 14650080 & TRUE      &                \\
16  & scc\_v100    & deepspeed      & 14617202 & TRUE      &                \\
17  & scc\_gtx1080 & profiler-torch & 14650758 & TRUE      & batch-size-64  \\
18  & scc\_gtx1080 & profiler-torch & 14650750 & TRUE      & sample-points  \\
19  & scc\_cpu     & profiler-torch & 14657599 & TRUE      & sample-points  \\
20  & scc\_gtx1080 & profiler-torch & 14650759 & TRUE      & batch-size-128 \\ \bottomrule
\end{tabular}
\caption{test caption}
\end{table}

footnote \footnote{\url{https://www.gwdg.de/web/guest/hpc-on-campus/scc} (Accessed on: 22.02.2023)}

\section{Results}
\label{sec:results}

\subsection{Original Workflow}
\label{sec:r-original-workflow}



\begin{figure}[H]
\centering
\includegraphics[width=0.5\textwidth]{./assets/sacct_barplot_by_nodes_sample-points-effect}
\caption*{}
\label{fig:sacct_barplot_by_nodes_sample-points-effect}
\end{figure}

\subsection{Runtime}
\label{sec:r-runtime}




\begin{figure}[H]
\centering
\includegraphics[width=1\textwidth]{./assets/sacct_barplot_by_nodes_no-experiment}
\caption*{Runtime on HPC}
\label{fig:sacct_barplot_by_nodes_no-experiment}
\end{figure}

\begin{figure}[H]
\centering
\includegraphics[width=1\textwidth]{./assets/sacct_barplot_by_nodes_no-experiment_gpu}
\caption*{Runtime on HPC (GPU-only)}
\label{fig:sacct_barplot_by_nodes_no-experiment_gpu}
\end{figure}

\subsection{PyTorch - Profiler}
\label{sec:r-pytorch-profiler}

\begin{figure}[H]
\centering
\includegraphics[width=0.7\textwidth]{./assets/scap_gtx1080_profiler-torch_14650076}
\caption*{Overview}
\label{fig:scap_gtx1080_profiler-torch_14650076}
\end{figure}

\begin{figure}[H]
\centering
\includegraphics[width=1\textwidth]{./assets/scap_gtx1080_profiler-torch_comparison-batch-size}
\caption*{Effect of increasing the batch size}
\label{fig:scap_gtx1080_profiler-torch_comparison-batch-size}
\end{figure}

\begin{figure}[H]
\centering
\includegraphics[width=1\textwidth]{./assets/scap_gtx1080_profiler-torch_batch-size-64_14650758_trace-view-laspy}
\caption*{Trace View: Laspy with presampled lidar data}
\label{fig:scap_gtx1080_profiler-torch_batch-size-64_14650758_trace-view-laspy}
\end{figure}

\begin{figure}[H]
\centering
\includegraphics[width=1\textwidth]{./assets/scap_gtx1080_profiler-torch_sample-points_14650750_trace-view-laspy}
\caption*{Trace View: Laspy with raw lidar data}
\label{fig:scap_gtx1080_profiler-torch_sample-points_14650750_trace-view-laspy}
\end{figure}



\subsection{Deepspeed - FlopsProfiler}
\label{sec:r-flopsprofiler}


\begin{listing}[H]
\inputminted[xleftmargin=1em,linenos,fontsize=\scriptsize, firstline=1,lastline=16]{python}{./assets/scap_gtx1080_deepspeed_14615344_4294967294_one-epoch.txt}
\caption{Summary}
\label{lst:scap_gtx1080_deepspeed_14615344_4294967294_one-epoch-summary}
\end{listing}

\begin{listing}[H]
\inputminted[xleftmargin=1em,linenos,fontsize=\scriptsize, firstline=18,lastline=31]{python}{./assets/scap_gtx1080_deepspeed_14615344_4294967294_one-epoch.txt}
\caption{Aggregated Profile per GPU}
\label{lst:scap_gtx1080_deepspeed_14615344_4294967294_one-epoch-aggregated}
\end{listing}

\begin{listing}[H]
\inputminted[xleftmargin=1em,linenos,fontsize=\tiny, firstline=33,lastline=48, breaklines]{python}{./assets/scap_gtx1080_deepspeed_14615344_4294967294_one-epoch.txt}
\caption{Detailed Profile per GPU}
\label{lst:scap_gtx1080_deepspeed_14615344_4294967294_one-epoch-detailed}
\end{listing}

\section{Discussion}
\label{sec:discussion}



\section{Conclusion}
\label{sec:conclusion}



% --- references ---

\newpage
\printbibliography[heading=bibintoc]

% --- your appendix ---
\appendix
\break

\pagenumbering{arabic}
\renewcommand*{\thepage}{A\arabic{page}}

\begin{figure}[H]
\centering
\includegraphics[width=1\textwidth]{./assets/scap_gtx1080_profiler-torch_batch-size-64_14650758_trace-view}
\caption*{Trace View}
\label{fig:scap_gtx1080_profiler-torch_batch-size-64_14650758_trace-view}
\end{figure}

\begin{figure}[H]
\centering
\includegraphics[width=1\textwidth]{./assets/scap_gtx1080_tensorboard_14615343}
\caption*{Tensorboard for run with GTX1080}
\label{fig:scap_gtx1080_tensorboard_14615343}
\end{figure}

\begin{figure}[H]
\centering
\includegraphics[width=0.85\textwidth]{./assets/sacct_barplot_by_nodes_batch-size-effect}
\caption*{result of increasing batch size on walltime}
\label{fig:sacct_barplot_by_nodes_batch-size-effect}
\end{figure}

\begin{figure}[H]
\centering
\includegraphics[width=0.7\textwidth]{./assets/scap_gtx1080_profiler-torch_batch-size-64_14650758}
\caption[test]{Overview: Increased batch size (64)}
\label{fig:scap_gtx1080_profiler-torch_batch-size-64_14650758}
\end{figure}

\begin{figure}[H]
\centering
\includegraphics[width=0.7\textwidth]{./assets/scap_gtx1080_profiler-torch_batch-size-128_14650759}
\caption[]{Overview: Increased batch size (128)}
\label{fig:scap_gtx1080_profiler-torch_batch-size-128_14650759}
\end{figure}

\begin{figure}[H]
\centering
\includegraphics[width=0.9\textwidth]{./assets/scap_gtx1080_profiler-torch_batch-size-64_14650758_operator-view}
\caption[Operator View]{}
\label{fig:scap_gtx1080_profiler-torch_batch-size-64_14650758_operator-view}
\end{figure}

\begin{figure}[H]
\centering
\includegraphics[width=0.9\textwidth]{./assets/scap_gtx1080_profiler-torch_batch-size-64_14650758_operator-view-details}
\caption[Operator View - Details]{}
\label{fig:scap_gtx1080_profiler-torch_batch-size-64_14650758_operator-view-details}
\end{figure}

\begin{figure}[H]
\centering
\includegraphics[width=0.9\textwidth]{./assets/scap_gtx1080_profiler-torch_batch-size-64_14650758_gpu-kernel-view}
\caption[GPU Kernel View]{}
\label{fig:scap_gtx1080_profiler-torch_batch-size-64_14650758_gpu-kernel-view}
\end{figure}

\begin{figure}[H]
\centering
\includegraphics[width=0.85\textwidth]{./assets/scap_gtx1080_profiler-torch_batch-size-64_14650758_memory-view}
\caption[Memory View]{}
\label{fig:scap_gtx1080_profiler-torch_batch-size-64_14650758_memory-view}
\end{figure}

\begin{figure}[H]
\centering
\includegraphics[width=1\textwidth]{./assets/scap_gtx1080_profiler-torch_batch-size-64_14650758_module-view}
\caption[Module View]{}
\label{fig:scap_gtx1080_profiler-torch_batch-size-64_14650758_module-view}
\end{figure}


\section{Code samples}

This is part of the appendix...


\end{document}  
