%pdflatex -shell-escape --output-directory=out report.tex
\documentclass[12pt, a4paper, hidelinks]{article}

\usepackage{graphicx}
\usepackage{hyperref}
\usepackage{xcolor}
\usepackage[english]{babel}
\usepackage[nottoc]{tocbibind}
\usepackage[utf8]{inputenc}
\usepackage[T1]{fontenc}
\usepackage[backend=biber,style=alphabetic]{biblatex}
\addbibresource{ref.bib} % The filename of the bibliography
\usepackage[left=2.5cm,right=2.5cm,top=2.5cm,bottom=2.5cm]{geometry}
\usepackage{acronym}
\usepackage{fancyhdr}
\pagestyle{fancy}
\usepackage{lipsum}
\usepackage{array, tabularx, booktabs} % For better tables
\usepackage{longtable}
%\usepackage{listings} % Code environments
%\usepackage{lstautogobble}  % Fix relative indenting
%\usepackage{color}          % Code coloring
%\usepackage{zi4}            % Nice font
\usepackage{minted}
\usemintedstyle{tango}
\setminted{linenos, autogobble, bgcolor=gray!5}
\usepackage{cleveref}
\usepackage{titlesec}
\setlength{\marginparwidth}{2cm}
\usepackage{todonotes}
\usepackage[autostyle=true]{csquotes}

\iffalse

\definecolor{codegreen}{rgb}{0,0.6,0}
\definecolor{codegray}{rgb}{0.5,0.5,0.5}
\definecolor{backcolour}{gray}{0.95}
\definecolor{codeorange}{rgb}{0.8,0.5,0.2}

\lstdefinestyle{mystyle}{
    backgroundcolor=\color{backcolour},  
    commentstyle=\color{codegray},
    keywordstyle=\color{codeorange},
    numberstyle=\tiny\color{codegray},
    stringstyle=\color{codegreen},
    basicstyle=\ttfamily\footnotesize,
    breakatwhitespace=false,         
    breaklines=true,                 
    captionpos=b,                    
    keepspaces=true,                 
    numbers=left,                    
    numbersep=5pt,                  
    showspaces=false,                
    showstringspaces=false,
    showtabs=false,                  
    tabsize=2
}

%\lstset{style=mystyle}

\definecolor{bluekeywords}{rgb}{0.13, 0.13, 1}
\definecolor{greencomments}{rgb}{0, 0.5, 0}
\definecolor{redstrings}{rgb}{0.9, 0, 0}
\definecolor{graynumbers}{rgb}{0.5, 0.5, 0.5}

\usepackage{listings}
\lstset{
    autogobble,
    columns=fullflexible,
    showspaces=false,
    showtabs=false,
    breaklines=true,
    showstringspaces=false,
    breakatwhitespace=true,
    escapeinside={(*@}{@*)},
    commentstyle=\color{greencomments},
    keywordstyle=\color{bluekeywords},
    stringstyle=\color{redstrings},
    numberstyle=\color{graynumbers},
    basicstyle=\ttfamily\footnotesize,
    frame=l,
    framesep=12pt,
    xleftmargin=12pt,
    tabsize=4,
    captionpos=b
}

\fi

\titleformat{\section}[hang]{\huge\bfseries}{\thesection\hspace{20pt}}{0pt}{\huge\bfseries}
\graphicspath{{Figures/}{./}{Assets/}}

% --- document configuration ---

\newcommand{\thesistitle}{[Put your title here]}  
\newcommand{\supervisor}{[Put your supervisors name here]} 
\newcommand{\authorname}{[Put your name here]}
\newcommand{\university}{Georg-August-Universität Göttingen}
\newcommand{\department}{Institute of Computer Science}
\newcommand{\thesistype}{Seminar Report}
\newcommand{\matrikelnumber}{[Put your matrikel number here]}
\newcommand{\keywords}{} % Set keywords that describe your report

\hypersetup{pdftitle=\thesistitle} % Set the PDF's title to your title
\hypersetup{pdfauthor=\authorname} % Set the PDF's author to your name
\hypersetup{pdfkeywords=\keywords} % Set the PDF's keywords to your keywords

\begin{document}

\fancyhead{}
\fancyhead[R]{\footnotesize \thesistitle}
\fancyfoot{}
\fancyfoot[R]{\thepage}
\fancyfoot[L]{Section \thesection}
\fancyfoot[C]{\authorname}
\renewcommand{\headrulewidth}{0.4pt}
\renewcommand{\footrulewidth}{0.4pt}

\pagestyle{plain}


    

% --- title page ---

\begin{titlepage}
\begin{minipage}[t]{0.6\textwidth}
\begin{flushleft}
\includegraphics[width=6.5cm]{logo-goettingen.pdf}
\end{flushleft}
\end{minipage}
\begin{minipage}[t]{0.4\textwidth}
\begin{center}
\qquad\includegraphics[width=2.5cm]{hps-logo.pdf}
\end{center}
\end{minipage}

\begin{center}

\vspace*{.06\textheight}
\LARGE \thesistype\\[0.5cm]

\rule{.9\linewidth}{.6pt} \\[0.4cm] % Horizontal line
{\huge \bfseries \thesistitle}\vspace{0.4cm}
\rule{.9\linewidth}{.6pt} \\[1.5cm] % Horizontal line
 
\Large\authorname\\
\hfill\\
\large MatrNr: \matrikelnumber\\ \vfill
Supervisor: \supervisor 
\vfill
\university\\
\department
\vfill
{\large \today}\\[4cm] % Date
 
\vfill
\end{center}
\end{titlepage}
    

% --- abstract ---

%\thispagestyle{empty}
\newpage
\pagenumbering{roman}
\setcounter{page}{1}

\section*{Abstract}


[Put your abstract here.]
Note that we typically publish your report as PDF on our webpage. Let us know if you disagree.]

\medskip

General structure of an abstract, write 1 to 2 sentences per section
\begin{enumerate}
\item  A general statement introducing the broad research area of the particular topic being investigated.
\item  An explanation of the specific problem (difficulty, obstacle, challenge) to be solved.
\item  A review of existing or standard solutions to this problem and their limitations.
\item  An outline of the proposed new solution.
\item  A summary of how the solution was evaluated and what the outcomes of the evaluation were.
\end{enumerate}

You may find the following resources useful:
\begin{itemize}
    \item \url{https://www.grammarly.com/blog/write-an-abstract/}
    \item \url{https://www.editage.com/insights/manuscript-structure-how-to-convey-your-most-important-ideas-through-your-paper}
    \item More useful links: \url{https://hps.vi4io.org/teaching/ressources/start}
\end{itemize}


%\thispagestyle{empty}
% --- table of contents ---
% Comment out the lists you are not using
\newpage


\clearpage
\phantomsection\pdfbookmark{\contentsname}{toc}
\tableofcontents

\newpage
\clearpage\phantomsection
\listoftables
%\newpage
%\clearpage
\phantomsection
\listoffigures
%\newpage
%\clearpage
\phantomsection
\listoflistings \addcontentsline{toc}{section}{List of Listings}

% if you have abbreviations
\newpage

\section*{List of Abbreviations} \addcontentsline{toc}{section}{List of Abbreviations}
\begin{acronym}[Bash] % Add acronyms such that they are shown in full only on first occurrence
    \acro{HPC}{High-Performance Computing}
\end{acronym}

\thispagestyle{plain}
\newpage

% --- content ---

\pagenumbering{arabic}
\setcounter{page}{1}
\pagestyle{fancy}


\section{Introduction}
[Any page limit starts with the main content here and ends with the biography (before the appendix).

The introduction is the most important part of such a report. Its general structure is similar to the Abstract but with about one paragraph per section as listed again below.
\begin{enumerate}
\item  A general statement introducing the broad research area of the particular topic being investigated.
\item  An explanation of the specific problem (difficulty, obstacle, challenge) to be solved.
\item  A review of existing or standard solutions to this problem and their limitations.
\item  An outline of the proposed new solution.
\item  A summary of how the solution was evaluated and what the outcomes of the evaluation were.
\end{enumerate}
Furthermore, the introduction should have a contributions section, that summarizes, possibly as a list, what the report and the work described in the report has contributed.
Finally, the introduction should end with an outline of the remaining report.

\ac{HPC} refers to the usage of powerful compute systems to solve non-trivial problems.

\subsection{Citation and figure example}
Hawthorn et al. \cite{20220610_dodge_measuring-the-carbon-intensity-of-ai-in-cloud-instances} talk about laser, which is similar to the work of his colleagues \cite{20220610_dodge_measuring-the-carbon-intensity-of-ai-in-cloud-instances, 20220610_dodge_measuring-the-carbon-intensity-of-ai-in-cloud-instances}.
\begin{figure}[th]
\centering
\includegraphics[width=0.7\textwidth]{speed_FILL0_wght400_GRAD0_opsz48.png}
\caption[An Electron]{An electron (artist's impression).}
\label{fig:Electron}
\end{figure}

\subsection{Table and listing example}
\Cref{tab:treatments} shows an example for a table in \LaTeX.
\begin{table}[th]
\label{tab:treatments}
\centering
\begin{tabularx}{0.45\textwidth}{l||r|r}
Groups & Treatment X & Treatment Y \\
\hline \hline
1 & 0.20 & 0.80\\
2 & 0.17 & 0.70\\
3 & 0.24 & 0.75\\
4 & 0.68 & 0.30\\
\end{tabularx}
\caption{The effects of treatments X and Y on the four groups studied.}
\end{table}

[If you show results in tables, always ensure all of them are aligned to the right and have the same precision.]

\Cref{lst:hello} shows an example of a listing, a good way to display code in \LaTeX.

\begin{listing}
\begin{minted}{Go}
package main
import "fmt"
func main() {
    fmt.Println("Hello, world!")
}
\end{minted}
\caption{"Hello, world!" in Go}
\label{lst:hello}
\end{listing}


\iffalse
\begin{lstlisting}[language=Go, caption={"Hello, world!" in Go}, label=lst:hello]
package main
import "fmt"
func main() {
    fmt.Println("Hello, world!")
}
\end{lstlisting}
\fi

\section{Conclusion}
\lipsum[2]



% --- references ---

\newpage
\printbibliography[heading=bibintoc]

% --- your appendix ---
\appendix
\break

\pagenumbering{arabic}
\renewcommand*{\thepage}{A\arabic{page}}



\section{Work sharing}

If you worked in a group, describe here how you distributed the work and the actual contributions of each peer.

\subsection{Hans}
...

\subsection{Peter}
...

\section{Code samples}

This is part of the appendix...


\end{document}  
